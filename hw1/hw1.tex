%        File: hw1.tex
%     Created: Wed Sep 14 04:00 PM 2016 C
% Last Change: Wed Sep 14 04:00 PM 2016 C
%

\documentclass[a4paper]{article}

\title{Homework 1 }
\date{9/21/16}
\author{Trevor Steil}

\usepackage{amsmath}
\usepackage{amsthm}
\usepackage{amssymb}

\newtheorem{theorem}{Theorem}[section]
\newtheorem{corollary}{Corollary}[section]
\newtheorem{proposition}{Proposition}[section]
\newtheorem{lemma}{Lemma}[section]
\newtheorem*{claim}{Claim}
\newtheorem*{problem}{Problem}
%\newtheorem*{lemma}{Lemma}
\newtheorem{definition}{Definition}[section]

\newcommand{\R}{\mathbb{R}}
\newcommand{\N}{\mathbb{N}}
\newcommand{\C}{\mathbb{C}}
\newcommand{\Q}{\mathbb{Q}}
\newcommand{\supp}[1]{\mathop{\mathrm{supp}}\left(#1\right)}
\newcommand{\lip}[1]{\mathop{\mathrm{Lip}}\left(#1\right)}
\newcommand{\curl}{\mathrm{curl}}
\newcommand{\la}{\left \langle}
\newcommand{\ra}{\right \rangle}
\renewcommand{\vec}[1]{\mathbf{#1}}

\newenvironment{solution}{\emph{Solution.}}

\begin{document}
\maketitle

\begin{enumerate}
  \item
    Let $X_1, X_2, \dots $ be random variables from $( \Omega, \mathcal{F}, P )$ to $\overline{ \R }$. Show that $\liminf_{n \to \infty} X_n$ is measureable.

    \begin{proof}

      Let $a \in \overline{\R}$. Define $X = \liminf_{n \to \infty} X_n$. If $\omega \in X^{-1}( [-\infty, a] )$, then for any $\varepsilon > 0$ and
      for any $n$, there is a $k > n$ such that $X_k(\omega) < a + \varepsilon$. That is
      \[ X^{-1} ( [- \infty, a] ) = \{ \omega : \forall n \ \exists k \geq n \text{ such that } X_k(\omega) \in [-\infty, a] \} \]

      We can also write
      \[ \bigcup_{k = n}^\infty X_k^{-1} ( [ -\infty, a ]) = \{ \omega : X_k(\omega) \in [-\infty, a] \text{ for some } k \geq n \} .\]
      Taking this another step, we get
      \[ \bigcap_{n=0}^\infty \bigcup_{k=n}^\infty X_k^{-1} ([-\infty, a]) = \{ \omega : \forall n \ \exists k \geq n \text{ such that } X_k(\omega) \in
      [-\infty, a] \} .\]

      That is,
      \[ X^{-1}( [-\infty, a] ) = \bigcap_{n=0}^\infty \bigcup_{k=n}^\infty X_k^{-1} ([-\infty, a]) \]

      Because each of the $X_k$ are measurable and $X^{-1}([-\infty, a])$ is expressed as a countable union and countable intersection of measurable
      sets, $X^{-1} ([-\infty, a])$ is measurable for all $a \in \overline{\R}$.

      The Borel $\sigma$-algebra is generated by open sets of the form $(a,b)$. We can write
      \[ (a,b) = \R \setminus [-\infty, a]^c \setminus \left( \R \setminus [-\infty,b] \right)^c \]
      so the sets of the form $[-\infty, a]$ also generate the Borel $\sigma$-algebra. Therefore, by Proposition 2.1, $X$ is measurable.

    \end{proof}

  \item
    Suppose that $F$ is a distribution function. Is it true that $F$ has only countably many jumps?

    \begin{claim}
      If $F$ is a distribution function, $F$ has at most countably many jumps.
    \end{claim}

    \begin{proof}

      Assume a distribution function $F$ has uncountably many jump discontinuities. Let $x_\alpha$ be one such point of discontinuity. Then for some
      $\varepsilon > 0$, we have
      \[ F(x_\alpha) - F(x_\alpha-\delta) > \varepsilon \]
      for any $\delta > 0$. By the density of $\Q$ in $\R$, we can find a rational $q_\alpha \in \left( F(x_\alpha) - \varepsilon, F(x_\alpha)
      \right)$. Because $F$ is increasing, there is no $x$ such that $F(x) = q_\alpha$.

      Repeating this process for every jump in $F$ would give a rational number between 0 and 1 for each of the uncountably many jumps in $F$. Because
      $F$ is increasing, each of these rationals would be unique (none of the jumps cover the same values of $F(x)$). Therefore, we would have a way of indexing $\Q$ by an uncountable set, which
      contradicts the countability of $\Q$. Thus, $F$ must only have countably many jumps.

    \end{proof}

  \item
    Suppose that $Y_n$ are random variables from $( \Omega, \mathcal{F}, P )$ to $\R$ such that $Y_n(\omega) \to Y(\omega)$ for each $\omega \in \Omega$. Show that $\forall \delta > 0, \ \exists$ a set $A$ with $P(A) < \delta$, so that $Y_n(\omega) \to Y(\omega)$ uniformly on $A^c$.

    \begin{proof}

      Let $\delta > 0$. For each pair of nonnegative integers, define
      \[ \Omega_k^n = \{ x \in \Omega : |Y_j(\omega) - Y(\omega)| < \frac{1}{n}, \text{ for all } j > k \} .\]

      Fix a value for $n$. We see that $\Omega_k^n \subset \Omega_{k+1}^n$ and $\Omega_k^n \to \Omega$ as $k \to \infty$. Therefore, we can find a
      $k_n$ such that $P(\Omega \setminus \Omega_{k_n}^n) < \frac{1}{2^n}$. We know that $\sum_{n=1}^\infty 2^{-n} = 1$, so we can choose $N$ such that
      $\sum_{n=N}^\infty 2^{-n} < \delta$.

      Define $A_\delta$ such that
      \[ A_\delta^c = \bigcap_{n\geq N} \Omega_{k_n}^n .\]

    Then
    \[ P( A_\delta ) \leq \sum_{n \geq N} P \left( \Omega \setminus \Omega_{k_n}^n \right) < \delta \]

    Let $\varepsilon>0$ and choose $n \geq N$ such that $\frac{1}{n} < \varepsilon$. Let $\omega \in A_\delta^c$. Then by definition, $\omega \in
    \Omega_{k_n}^n$. Therefore, $|Y_j(\omega) - Y(\omega)| < \varepsilon$ for any $j > k_n$. Therefore $Y_n(\omega) \to Y(\omega)$ uniformly on
    $A_\delta^c$.

    \end{proof}

  \item
    A probability space $ ( \Omega, \mathcal{F}, P )$ is said to be nonatomic if for each $A \in \mathcal{F}$ with $P(A) > 0$, there exists $B \subset A$ with $0 < P(B) < P(A)$. Show that, for each nonatomic probability measure, there is a $C \in \mathcal{F}$ with $P(C) = \frac{1}{2}$.

    \begin{proof}

      First, we will show that for any $n \in \N$, we can find a set $E$ such that $P(E) \in [2^{-n-1}, 2^{-n})$.

      If $n=0$, we can find a set $E \subset \Omega$ such that $0 < P(E) < P(\Omega) = 1$ because our probability space is nonatomic. Then either $\frac{1}{2} \leq P(E) < 1$ or
      $\frac{1}{2} \leq P(E^c) < 1$.

      Now assume we have a set $E \subset \Omega$ such that $P(E) \in [2^{-n}, 2^{-n+1})$ for some $n \in \N$. By nonatomicity, we can find a set
      $E_1 \subset E$ such that $0 < P(E_1) < P(E)$. Then either, $P(E_1) \in [2^{-n-1}, 2^{-n+1})$. (One set must be at least half the size of
      $E$, but the larger set could be almost as large as $E$.)

      Assume $E_1$ is the larger of the two sets. If $P(E_1) \in [2^{-n-1}, 2^{-n})$, we are done.
        If $P(E_1) \in [2^{-n}, 2^{-n+1})$, we can continue
        taking subsets ${E_1 \supset E_2 \supset \dots}$ until $2^{-n-1} < P(E_k) < 2^{n}$.

        Assume such a $k$ doesn't exist. Let
        \[ p = \inf \{ P(B) : B \subset E_1, P(B) > 2^{-n-1} \} > 2^{-n} \]

        We can find a sequence $E_1 \supset E_2 \supset \dots$ such that $P(E_i) \to p$. Define
        \[ A = \bigcap_{i=1}^\infty E_i .\]
        $A$ is measurable because it is the countable intersection of measurable sets. Also, by Proposition 3.1,
        \[ P(A) = \lim_{i \to \infty} P(E_i) = p .\]
        Using nonatomicity, we can find a set $A' \subset A$ such that $2^{-n-1} < P(A') < P(A)$. Therefore, $A' \in \{ B \subset E_1, P(B) > 2^{-n-1}
      \}$ and $P(A') < p$, which is a contradiction. Therefore, by taking further subsets, we can find a set $E_k$ such that $P(E_k) \in [2^{-n-1},
        2^{-n})$. So by induction, for any $n \in \N$ we can find a set $E \in \Omega$ such that $2^{-n-1} \leq P(E) < 2^{-n}$

        By modifying the argument above slightly, we see that we can find sets of arbitrarily small measure inside of any set of positive measure. Now
        we will find our set $C$ with $P(C) = \frac{1}{2}$. First, take $C_1 \in \Omega$ such that $\frac{1}{4} \leq P(C_1) < \frac{1}{2}$. Let $r_1 =
        \frac{1}{2} - P(C_1) \leq \frac{1}{4}$.

        Now we find smaller and smaller sets to fill in the missing measure. For simplicity, we assume in the following that we never add a set that
        makes the total measure above $\frac{1}{2}$. If this were to happen, we would take a subset within the sets we have already found and take it
        out of our union to get closer to a total measure of $\frac{1}{2}$. Assume that we have disjoint sets $C_1, \dots, C_n$ such that
        \begin{align*}
          r_n &= \frac{1}{2} - P \left( \bigcup_{i=1}^n C_i \right) \\
          &= \frac{1}{2} - \sum_{i=1}^n P(C_i)
        \end{align*}
        satisfies $0 \leq r_n < 2^{-n}$. If $r_n = 0$, then $C = \bigcup_{i = i}^n C_i$ is the desired set. Otherwise, let $k$ be the
        integer such that $2^{-k} \leq r_n < 2^{-k+1}$. By assumption, $2^{-k+1} \leq 2^{-n}$. By what we already showed, we can find a set $C_{n+1} \in \Omega
        \setminus \bigcup_{i=1}^n C_i$ such that $P(C_{n+1}) \in [2^{-k}, 2^{-k+1})$. Then we have
          \begin{align*}
            r_{n+1} &= r_{n} - P(C_{n+1}) \\
            &< 2^{-k+1} - 2^{-k} \\
            &\leq 2^{-k} \\
            &\leq 2^{-(n+1)}
          \end{align*}

          Thus $P(\bigcup_{i=1}^n C_i) \to \frac{1}{2}$ as $n \to \infty$. Therefore, we get a set $C = \bigcup_{i=1}^\infty C_i$ which is measurable
          because it is a countable union of measurable sets, and $P(C) = \lim_{n \to \infty} P( \bigcup_{i=1}^n C_i ) = \frac{1}{2}$

    \end{proof}

\end{enumerate}

\end{document}


