%        File: hw2.tex
%     Created: Wed Oct 05 04:00 PM 2016 C
% Last Change: Wed Oct 05 04:00 PM 2016 C
%

\documentclass[a4paper]{article}

\title{Math 8651 Homework 2 }
\date{10/17/16}
\author{Trevor Steil}

\usepackage{amsmath}
\usepackage{amsthm}
\usepackage{amssymb}
\usepackage{esint}
\usepackage{bbm}

\newtheorem{theorem}{Theorem}[section]
\newtheorem{corollary}{Corollary}[section]
\newtheorem{proposition}{Proposition}[section]
\newtheorem{lemma}{Lemma}[section]
\newtheorem*{claim}{Claim}
\newtheorem*{problem}{Problem}
%\newtheorem*{lemma}{Lemma}
\newtheorem{definition}{Definition}[section]

\newcommand{\R}{\mathbb{R}}
\newcommand{\N}{\mathbb{N}}
\newcommand{\C}{\mathbb{C}}
\newcommand{\Z}{\mathbb{Z}}
\newcommand{\E}{\mathbb{E}}
\newcommand{\supp}[1]{\mathop{\mathrm{supp}}\left(#1\right)}
\newcommand{\lip}[1]{\mathop{\mathrm{Lip}}\left(#1\right)}
\newcommand{\curl}{\mathrm{curl}}
\newcommand{\la}{\left \langle}
\newcommand{\ra}{\right \rangle}
\renewcommand{\vec}[1]{\mathbf{#1}}

\newenvironment{solution}{\emph{Solution.}}

\begin{document}
\maketitle

\begin{enumerate}
  \item
    \begin{problem}

      Let $\{A_n\}$ be a countable collection of measurable sets with respect to the probability space $(\Omega, \mathcal{F}, P)$, and assume that
      there exists $\delta > 0$ such that $P(A_n) > \delta$ for all $n$. True or false: there is at least one point in $\Omega$ which belongs to
      infinitely many $A_n$.

    \end{problem}

    \begin{solution}
      Define the random variables
      \[ X_n = \mathbbm{1}_{A_n} \]
      for all $n$. Let
      \[ X = \sum_{i=1}^\infty X_i .\]
      The partial sums $\sum_{i=1}^N X_i$ are an increasing sequence of random variables. Therefore,
      \begin{align*}
        \E[X] &= \lim_{N \to \infty} \sum_{i=1}^N \E[X_i] \\
        &> \lim_{N \to \infty} \sum_{i=1}^N \delta \\
        &= \lim_{N \to \infty} N \delta \\
        &\to \infty
      \end{align*}

      Therefore, there is a set $B \subset \Omega$ such that $P(B) > 0$ and $X(\omega) = \infty$ for all $\omega \in B$.

    \end{solution}

  \item
    \begin{problem}

      Assume $X \geq 0$, where $X$ is a real valued random variable. Is $\sigma^2(X \wedge y)$ an increasing function of $y$?

    \end{problem}

    \begin{solution}

    \end{solution}

  \item

    \begin{problem}

      Let $X_1, X_2, \dots$ be real valued random variables on the same probability space $(\Omega, \mathcal{F}, P)$, with density functions
      \[ f_{X_n}(x) =
        \begin{cases}
          e^n &\text{on } \left( 0, e^{-n} \left( 1 - \frac{1}{n} \right) \right) \\
          \frac{1}{x^2} &\text{on } (n, \infty)
        \end{cases}
      \]
      and assume that $X_1, X_2, \dots$ are independent. What can you say about $S_n = X_1 + \dots + X_n$ as $n \to \infty$?

    \end{problem}

    \begin{solution}

    \end{solution}

  \item

    \begin{problem}

      If $X_1, X_2,\dots$ is a sequence of identically distributed real valued random variables on the same probability space, with finite
      expectations, then is it necessarily the case that
      \[ \lim_{n \to \infty} \frac{1}{n} \mathbb{E} \left[ \max_{1 \leq j \leq n} |X_j| \right] = 0 ? \]

    \end{problem}

    \begin{solution}

    \end{solution}
\end{enumerate}
\end{document}


