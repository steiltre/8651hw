%        File: hw3.tex
%     Created: Wed Nov 02 01:00 PM 2016 C
% Last Change: Wed Nov 02 01:00 PM 2016 C
%

\documentclass[a4paper]{article}

\title{Math 8651 Homework 3 }
\date{11/7/16}
\author{Trevor Steil}

\usepackage{amsmath}
\usepackage{amsthm}
\usepackage{amssymb}
\usepackage{esint}
\usepackage{enumitem}
\usepackage{algorithm}
\usepackage{algorithmicx}
\usepackage{algpseudocode}

\newtheorem{theorem}{Theorem}[section]
\newtheorem{corollary}{Corollary}[section]
\newtheorem{proposition}{Proposition}[section]
\newtheorem{lemma}{Lemma}[section]
\newtheorem*{claim}{Claim}
\newtheorem*{problem}{Problem}
%\newtheorem*{lemma}{Lemma}
\newtheorem{definition}{Definition}[section]

\newcommand{\R}{\mathbb{R}}
\newcommand{\N}{\mathbb{N}}
\newcommand{\C}{\mathbb{C}}
\newcommand{\Z}{\mathbb{Z}}
\newcommand{\Q}{\mathbb{Q}}
\newcommand{\supp}[1]{\mathop{\mathrm{supp}}\left(#1\right)}
\newcommand{\lip}[1]{\mathop{\mathrm{Lip}}\left(#1\right)}
\newcommand{\curl}{\mathrm{curl}}
\newcommand{\la}{\left \langle}
\newcommand{\ra}{\right \rangle}
\renewcommand{\vec}[1]{\mathbf{#1}}

\newenvironment{solution}[1][]{\emph{Solution #1}}

\algnewcommand{\Or}{\textbf{ or }}
\algnewcommand{\And}{\textbf{ or }}

\begin{document}
\maketitle

\begin{enumerate}
  \item
    \begin{problem}

      Let $X_k$ be i.i.d. binomial random variables of the form $P(X_k = j) = \binom{10}{j} \left( \frac{1}{3} \right)^j \left( \frac{2}{3}
      \right)^{10-j}$. What is $\lim_{n \to \infty} \frac{T_n}{n}$, where $T_n = \sum_{k=1}^n (X_k)^2$? (Do not use any facts except those from class,
      although you may write a binomial random variable as a sum of other random variables.)

    \end{problem}

    \begin{solution}

    \end{solution}

  \item
    \begin{problem}

      Let $\omega = . \omega_1 \omega_2 \omega_3 \dots$ be the decimal representation of $\omega \in [0,1]$. The triple 4,4,5 is said to occur in
      position $k$ of $\omega$ if $\omega_k = \omega_{k+1} = 4$ and $\omega_{k+2} = 5$. Let $Y_n$ be the total number of times this triple occurs up
      to position $n$. What can you say about the limiting behavior of $Y_n(\omega)$ for a typical $\omega$? (This problem is posed somewhat vaguely
      -- you should fill in the details after deciding what to show.)

    \end{problem}

    \begin{solution}

    \end{solution}

  \item
    \begin{problem}
      Let $X_1, X_2, \dots$ be i.i.d. random variables with $P(X_1 = 2^k) = 2^{-k}, k \geq 1$. Show that for appropriate $c_n$, $\frac{S_n - c_n}{n
      \log n} \to 0$ in probability, where $S_n = \sum_{k=1}^n X_k$.
    \end{problem}

    \begin{proof}

    \end{proof}

  \item
    \begin{problem}
      Let $S_n$ be chosen as in the previous problem. Show that there is no choice of $c_n$ so that $\frac{S_n - c_n}{n} \to 0$ in probability.
    \end{problem}

    \begin{proof}

    \end{proof}

\end{enumerate}
\end{document}


