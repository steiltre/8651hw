%        File: hw4.tex
%     Created: Thu Dec 01 01:00 PM 2016 C
% Last Change: Thu Dec 01 01:00 PM 2016 C
%

\documentclass[a4paper]{article}

\title{Math 8651 Homework 4 }
\date{12/7/16}
\author{Trevor Steil}

\usepackage{amsmath}
\usepackage{amsthm}
\usepackage{amssymb}
\usepackage{esint}
\usepackage{enumitem}
\usepackage{algorithm}
\usepackage{algorithmicx}
\usepackage{algpseudocode}
\usepackage{bbm}

\newtheorem{theorem}{Theorem}[section]
\newtheorem{corollary}{Corollary}[section]
\newtheorem{proposition}{Proposition}[section]
\newtheorem{lemma}{Lemma}[section]
\newtheorem*{claim}{Claim}
\newtheorem*{problem}{Problem}
%\newtheorem*{lemma}{Lemma}
\newtheorem{definition}{Definition}[section]

\newcommand{\R}{\mathbb{R}}
\newcommand{\N}{\mathbb{N}}
\newcommand{\C}{\mathbb{C}}
\newcommand{\Z}{\mathbb{Z}}
\newcommand{\Q}{\mathbb{Q}}
\newcommand{\E}{\mathbb{E}}
\newcommand{\supp}[1]{\mathop{\mathrm{supp}}\left(#1\right)}
\newcommand{\lip}[1]{\mathop{\mathrm{Lip}}\left(#1\right)}
\newcommand{\curl}{\mathrm{curl}}
\newcommand{\la}{\left \langle}
\newcommand{\ra}{\right \rangle}
\renewcommand{\vec}[1]{\mathbf{#1}}

\newenvironment{solution}[1][]{\emph{Solution #1}}

\algnewcommand{\Or}{\textbf{ or }}
\algnewcommand{\And}{\textbf{ or }}

\begin{document}
\maketitle
\begin{enumerate}
  \item
    \begin{problem}
      Let $X_k, Y_k, k=1,2, \dots$ and $X,Y$ be random variables on the same probability space $(\Omega, F, P)$.
      \begin{enumerate}
        \item If $X_k \to X$ in distribution and $Y_k \to Y$ in distribution as $k \to \infty$, is it necessarily true that $(X_k, Y_k) \to (X,Y)$ in
          distribution? (Interpret this to mean $\E [ f(X_k, Y_k) ] \to E[ f(X,Y) ]$ as $k \to \infty$ for all bounded, continuous $f: \R^2 \to \R$.)

        \item If $X_k \to X$ and $Y_k \to Y$ in probability, is it necessarily true that $(X_k,Y_k) \to (X,Y)$ in probability?

      \end{enumerate}

    \end{problem}

    \begin{solution}

      \begin{enumerate}
        \item
          Let $f: \R^2 \to \R$ be a bounded, continuous function. Let $F_k$, $F$, $G_k$, and $G$ be the distribution functions for $X_k, X, Y_k$ and $Y$, respectively.
          Each of these measures is finite. Therefore, because $f$ is bounded, we have
          \[ \int_{-\infty}^{\infty} \int_{-\infty}^{\infty} |f(x,y)| (F_n \times G_k)(dx \times dy) < \infty \]
          by Tonelli's Theorem. Therefore, we can use Fubini's Theorem to get for any fixed $k$
          \begin{align*}
            \lim_{n \to \infty} \int_{-\infty}^{\infty} \int_{-\infty}^{\infty} f(x,y) (F_k \times G_n)(dx \times dy) &= \int_{-\infty}^{\infty} f(x,y)
            F_k(dx) G(dy)
          \end{align*}
          and similarly with $k$ and $n$ switched and $F_k$ and $G_k$ replaced with $F$ and $G$. Therefore,
          \[ \lim_{n \to \infty} \int_{\R^2}^{} f(x,y) (F_n \times G_n) (dx \times dy) = \int_{\R^2}^{} f(x,y) (F \times G) (dx \times dy) \]
          and $(X_k, Y_k) \to (X,Y)$ in distribution.

        \item
          Assume $X_n \to X$ and $Y_n \to Y$ in probability. We are interested in $\{ |(X_n, Y_n) - (X,Y)| > \varepsilon \}$. If $|(X_n, Y_n) - (X,Y)| > \varepsilon$,
          then $|X_n - X| > \frac{\sqrt{2}}{2} \varepsilon$ or $|Y_n - Y| > \frac{\sqrt{2}}{2} \varepsilon$. This can be proved by contradiction. If
          $|X_n-X|, |Y_n-Y| \leq \frac{\sqrt{2}}{2} \varepsilon$, then
          \begin{align*}
            |(X_n, Y_n) - (X,Y)| &= \sqrt{|X_n-X|^2 + |Y_n-Y|^2} \\
            &\leq \sqrt{\frac{1}{2} \varepsilon^2 + \frac{1}{2} \varepsilon^2} \\
            &= \varepsilon
          \end{align*}
          Therefore, if $|(X_n, Y_n) - (X, Y)| > \varepsilon$, we must have $|X_n - X| > \frac{\sqrt{2}}{2} \varepsilon$ or $|Y_n - Y| > \frac{\sqrt{2}}{2}
          \varepsilon$.

          This means
          \[ \{ |(X_n, Y_n) - (X,Y)| > \varepsilon \} \subseteq \left( \{ |X_n - X| > \frac{\sqrt{2}}{2} \varepsilon \} \times \Omega \right) \cup
            \left( \{ |Y_n - Y| > \frac{\sqrt{2}}{2} \varepsilon \} \times \Omega \right) .\]

          If we take probabilities, we have
          \[ P \left( \{ |(X_n, Y_n) - (X,Y)| > \varepsilon \} \right) \leq P \left( \{ |X_n - X| > \frac{\sqrt{2}}{2} \varepsilon \} \right) +
          P \left( \{ |Y_n - Y| > \frac{\sqrt{2}}{2} \varepsilon \} \right) \]
          because $P(\Omega) = 1$ (we have abused notation slightly here and are using $P$ for the measure on our original probability space as well
          as the product space). Both terms on the right go to 0 as $n \to \infty$ because $X_n \to X$ and $Y_n \to Y$ in probability, so
          \[ \lim_{n \to \infty} P \left( \{ |(X_n, Y_n) - (X,Y)| > \varepsilon \} \right) = 0, \]
          and $(X_n, Y_n) \to (X,Y)$ in probability.

      \end{enumerate}

    \end{solution}

  \item
    \begin{problem}
      Show that the characteristic function of the Poisson distribution with parameter $\lambda$ is $\exp \left( \lambda( e^{it} - 1) \right)$.
    \end{problem}

    \begin{proof}

      Let $X$ be a random variable with a Poisson distribution. We know the probability mass function of $X$ is given by $\frac{\lambda^k
        e^{-\lambda}}{k!}$. Therefore, we can compute
        \begin{align*}
          \varphi(t) &= \E [ e^{itX} ] \\
          &= \sum_{k=0}^\infty e^{itk} \frac{\lambda^k e^{-\lambda}}{k!} \\
          &= e^{-\lambda} \sum_{k=0}^\infty \frac{\left( \lambda e^{it} \right)^k}{k!} \\
          &= e^{-\lambda} \exp \left( \lambda e^{it} \right) \quad \parbox{5cm}{by the power series for $e^x$} \\
          &= \exp \left( \lambda \left( e^{it} - 1 \right) \right)
        \end{align*}

    \end{proof}

  \item
    \begin{problem}
      Suppose that $X$ and $Y$ are independent random variables and that $X$ and $X+Y$ have the same distribution. What can you say about $Y$?
      Explain.
    \end{problem}

    \begin{solution}
      Because $X$ and $X+Y$ have the same distribution. They have the same characteristic function as well. Therefore,
      \begin{align*}
        \varphi_X(t) &= \varphi_{X+Y} (t) \\
        &= \varphi_X(t) \varphi_Y(t) \quad \parbox{5cm}{by independence}
      \end{align*}
      for all $t$. Therefore, we have
      \[ 0 = \varphi_X(t) \left( 1 - \varphi_Y(t) \right) .\]

      $\varphi_X$ is a characteristic function, so $\varphi(0) = 1$ and $\varphi$ is continuous. In particular, there is a neighborhood $U$ of 0 such
      that $\varphi(t) \neq 0$ for all $t \in U$. Therefore, $\varphi_Y(t) = 1$ for all $t \in U$.

      By definition, $\varphi_Y(t) = \int_{\Omega}^{} e^{itY} dP$. The only way for this
      to equal $1$ for all $t \in U$ is for $Y=0$ almost surely. Thus, for $X$ and $X+Y$ to have the same distribution, $Y=0$ almost surely.
    \end{solution}

  \item
    \begin{problem}
      Suppose that $\varphi(t)$ is a characteristic function. Is it necessarily true that $|\varphi(t)|$ is a characteristic function? (Hint: consider
      $\cos t$.
    \end{problem}

    \begin{solution}

      Consider the discrete random variable with $P(X=-1) = P(X=1) = \frac{1}{2}$. Then
      \begin{align*}
        \E [ e^{itX} ] &= \int_{\R}^{} e^{itX} dP \\
        &= \frac{1}{2} \left( e^{-it} + e^{it} \right) \\
        &= \cos t
      \end{align*}

      Therefore, $\cos t$ is the characteristic function for this random variable.

      Now we must show $| \cos t |$ is not a characteristic function. Assume it is. Let $\mu$ be the associated measure. By Theorem 6.2.4 in Chung, we
      have
      \[ \mu(x) = \lim_{T \to \infty} \frac{1}{2T} \int_{-T}^{T} e^{-itx} | \cos t | dt .\]
      Because $\cos t$ and $e^{-itx}$ are both $2 \pi$ periodic, we can simplify this to
      \[ \mu(x) = \frac{1}{2 \pi} \int_{0}^{2\pi} e^{-itx} | \cos t | dt .\]

      We recognize this as the expression for the $x^{th}$ Fourier coefficients for $| \cos t |$. We can write the Fourier series as a cosine series because $| \cos t |$ is an even
      function. The series that we get is
      \[ | \cos t | = \frac{2}{\pi} + \frac{4}{\pi} \sum_{m=1}^\infty \frac{(-1)^m}{1 - 4m^2} \cos(2mt) . \]

      We now need to find a negative coefficient in this series. By looking at the second term (and noticing our series had odd terms drop out), we
      get
      \begin{align*}
        \mu(4) &= \frac{4}{\pi} \frac{(-1)^2}{1 - 4 * 2^2} \\
        &= - \frac{4}{15 \pi}
      \end{align*}
      which contradicts $\mu$ being a measure.

    \end{solution}

  \item
    \begin{problem}
      The Central Limit Theorem states that under appropriate conditions on IID random variables $X_1, X_2, \dots, \frac{S_n}{\sqrt{n}} \to Z$ weakly,
      where $Z$ is $N(0,1)$. Why or why not can this convergence be in probability or almost surely?
    \end{problem}

    \begin{solution}

    \end{solution}

\end{enumerate}
\end{document}


